\documentclass{beamer}	
\mode<presentation>
 
\usepackage{pdfpages}
\usepackage{fancyvrb}
\usepackage{chemarr}

\usepackage{amsmath}		%% mathematics typesetting
\usepackage{amssymb}
 
\usepackage{epigraph}   %% nice setting of quotations

\usepackage{tabularx} %% allows to use row colours in tables

\usepackage{ulem}

\usepackage{booktabs}

\usepackage{siunitx} %% tpyeset SI units

\usepackage{CJKutf8} %% typeset Chinese characters

\usepackage{pdfpages}%% include pdfs


\usepackage{animate} %% show animated gifs

\DeclareMathAlphabet{\mathcalligra}{T1}{calligra}{m}{n}


% Color and Theme. Can be changed. However, this one's quite nice.
\usetheme{Madrid}
\definecolor{theme}{rgb}{0.84,0,0.21}
\usecolortheme[named=theme]{structure}


%%  Title information
\title[M11.11.1 Vegetatives Nervensystem I]{M11.11.1 Vegetatives Nervensystem 1}
\author[melanie.stefan@medicalschool-berlin.de]{}
\institute[]{Prof. Melanie Stefan - melanie.stefan@medcialschool-berlin.de}
\date{SoSe 2022}
 

% Table of contents to pop up at the beginning of each section
\AtBeginSection[]
{
  \begin{frame}<beamer>
    \frametitle{Outline}
    \tableofcontents[currentsection,currentsubsection]
  \end{frame}
}
 
\beamertemplatenavigationsymbolsempty

\begin{document}


{ \usebackgroundtemplate{\includegraphics[width=1.2\paperwidth]{MSB_Titelseite.pdf}} 
\begin{frame}

 \maketitle 

$\,$\\[6cm]


\end{frame} 
}



%% Hook: 
\begin{frame}
\begin{center}
\includegraphics[width=0.8\textwidth]{bauchhirn.png}    
\end{center}

\end{frame}

%% %% TLIA
{ \usebackgroundtemplate{ \includegraphics[width=1.2\textwidth]{achterbahn.jpg}     }
\begin{frame}
\frametitle{In dieser Vorlesung geht es um \dots }

$\;$\\[5cm]

\dots das vegetative Nervensystem

\end{frame}
}
%% %% Learning Objectives
 
\begin{frame}

 \frametitle{Nach dieser Vorlesung sollten Sie folgendes können}



\begin{block}{Grundlagen:}
\begin{itemize}
\item

\end{itemize}

\end{block}

\begin{block}{Klinik:}
\begin{itemize}
\item


\end{itemize}

\end{block}


\end{frame}




%% %% %% Main Body


\section{Vegetatives Nervensystem (VNS): Allgemeines }


% ************************
\begin{frame}{Vegetatives Nervensystem: Allgemeines}

\begin{itemize}
    \item 
auch: autonomes Nervensystem, viszerales Nervensystem
\item
zentrale und periphere Komponenten
\item
bekommt Anweisungen vom limbischen System
\end{itemize}

\end{frame}

% ***********************

\begin{frame}{Vegetatives Nervensystem: Allgemeines}

\begin{center}
\includegraphics[width=\textwidth]{brain_functions.png}    
\end{center}


\end{frame}



% ***********************

\begin{frame}{Vegetatives Nervensystem: Allgemeines}

\begin{itemize}
    \item 
    nicht willentlich beeinflussbar
    \item
    (schnelle) Anpassung vegetativer Funktionen des Körpers an aktuelle Erfordernisse, Koordination von Organfunktion: essenziell für Überleben
\end{itemize}



\end{frame}


\begin{frame}{Funktionen des vegetativen Nervensystem}

\begin{itemize}
    \item
    Unterstützung motorischer Leistung
    \item
    Körperhomöostase  (Aufrechterhaltung und Anpassung physiologischer Parameter)
    \item
    Vegetative Reflexe
    \item
    beeinflusst Aktivität von:
    \begin{itemize}
        \item 
        Erfolgsorganen  - Regulierung glatte Muskulatur
        \item
        Herzmuskulatur
        \item
        Drüsen und sekretorische Gewebe
    \end{itemize}
   
\end{itemize}

\end{frame}



\section{Komponenten des vegetativen Nervensystems}


% Vegetatives Nervensystem: Komponenten

% - Zentral und peripher

% - Peripher:
%     - Sympathikus (efferent)
%         - Nebennierenmark
%     - Parasympathikus (efferent)
%         - Enterisches Nervensystem
%     - Viszerale Afferenzen

% Symphathikus und Parasympathiskus koennen dieselben Zielorgane innervieren und synergistisch oder antagonistisch wirken


% [Abbildung: ZNS, Ganglion, Erfolgsorgan]

% [13:02]

% ************************

% Vegetatives Nervensystem: Komponenten


% Unterschiede in Organisation

% \begin{tabular}{ll}
% Sympathikus &    Parasympathikus \\
% Praeganglionaerer zellkoerper im Brust- und Lendenmark   &   Praegangionaerer Zellkoerper im Hirnstamm und Sakralmark \\
% Ganglien organfern: prae- und paravertebral & Ganglien organnah und z.T. in Erfolgsorgan \\
% \end{tabular}

% Transmitter und Rezeptoren

% Sympathikus &    Parasympathikus \\
% Praeganglionaerer Transmitter: ACh & Praeganglionaerer Transmitter: ACh \\
% Rezeptor: nAChR (Ionenkanal) & Rezeptor: nAChR (Ionenkanal) \\
% Postganglionaerer Transmitter: NA* & Postgangionaerer Transmitter: ACh \\
% Rezeptor: adrenerg (metabotrop) & Rezeptor: muskarinisch (metabotrop)

% * bis auf Schweissdruese: ACh -> muskarinischer Rezeptor

% ************************ [16:34]

% \begin{frame}
% \frametitle{Vegetatives Nervensystem: Komponenten}


% paravertebrale sympathische + parasympathische Ganglien: Divergenz -> Verbreitung von Informationen

% praevertebrale sympathische Ganglien: Konvergenz
% -> Integration von Informationen
% -> erhoeht Uebertragungserfolg

% [Picture Divergenz Konvergenz]

% Je mehr praeganglionaere Neurone ein postganglionaeres Neuron synaptisch innervieren, desto wahrscheinlicher ist es, dass eine Depolarisation erreicht werden kann, die positiv genug ist, um ein Aktionspotential auszuloesen

% Relais-Funktion: nur INformationsfranser, kann durch Konvergenz erreicht werden (paravertebrale sympathische Ganglien)


% \end{frame}

% ************************ [19:31]

% \begin{frame}{Vegetatives Nervensystem: Komponenten - Nebennierenmark}

% = Sympathisches postganglionaeres Paraganglion -> Umschaltung auf Hormon


% [Abbildung Anatomie: Nebenniere eben niere, mark und Rinde, Chromaffinzelle mit Ca Kanal, Adrenalin, nAChR Sympathikus mit ACh, Blutstrom 80% Adrenalin, 20% Noradrenalin*****]

% ***** check that thats right

% Nebennierenmark: aus Chromaffinzellen = endokrin wirksame postganglionaere Neurone = neuroendokrin

% Adrenalin: Stffwechselhormon, Stimulierung metabolischer Prozesse

% \end{frame}

% ************************ [23:59]

% \begin{frame}{}
% Vegetatives Nervensystem: Komponenten - ES

% [Abbildung ENS MDT (Magen-Darm-Trakt), postganglionaer sympathisch, praeganglionare parasympathisch]

% ENS: 
% = Spezielles Nervenszsystem des Magen-Darm-Trakts
% - koordiniert Kontration Magen-Darm-Muskulatur
% - kann ZNS-unabhaenging funktionieren
% - funktionell parasympathisches Ganglion

% Funktionen enterisches Nervensystem (ENS)
% 1. Koordination Kontraktion Muskulature MAgen-Darm-Trakt 
% -> unidirektionaler Transport von Oesophagus zu Anus

% 2. IOnentransport
% -> Sekretion und Absoprtion
% 3. Regulation lokale Blutzufuhr


% ENS ist in Lage, adequat auf lokale Stimul zu reagieren. Sympathikus und Parasympathikus greifen regulierend ein wenn noetig
% \end{frame}

% ************************ [27:05]

% Komponenten ENS:
% Plexus myentericus (Auerbach)
% zwischen Ring- und Laengsmuskelschikt Muskularis externa -> Motorok MDT

% Plexus submucosus (Mei\ss ner)
% Submucosa des Duenn- und Dickdarms
% -> epitheliale Transportprozesse

% Innervierung durch VNS:
% Sympathikus:
% -> praevertebrale Ganglien
% -> distales Kolon und Rektum*

% Parasympathikus:
% -> N. vagus: Oesophagus bis proximales Kolon
% -> sakrale Segmente: distales Kolon und Rektum*



% *Sympathikus innterviert hier v.a. enterale Ganglienzellen udn glatte Muskulatur Blutgefaesse
% Parasympathikus innerviert hier v.a. Mukosa und Ringmuskulatur

% ************************ [28:54]

% Vegetatives Nervensystem: Komponenten - ENS


% [Abbildung Effekte sympathisch, parasympathisch]

% Hirschsprungkrankheit:
% - ENS in Abschnitt Dickdarm nicht voll ausgebildet, da Vorlaeuferzellen nicht gesamte Laenge eingewadnert sind -> Verstopfung und Symptome Darmverschluss (keine Muskelkontraktion)

% Verdacht liegt vor, wenn Neugeborenes nicht binnen 24h Mekonium ausscheidet -> chirurgische Entfernung betroffener Darmanteil

% ************************ 

% \begin{frame}{Frame Title}
    

% Vegetatives Nervensystem: Komponenten - 
% Viszerale Afferenzen

% [Abbildung ZNS - viserale Afferenzen, (para) sympathische Efferenzen - Erfolgsorgan plus Pfeil von Geruch/Geschmack nach ZNS]

% Allgemeine viszerale Afferenzen (AVA)
% Chemo-, mehano- und Schmerzrezeptoren -> mechanische, thermische, metbolische und entzuendliche Zustaende

% Spezielle viszerale Afferenzen (SVA)
% Geruchsrezeptoren, Geschmacksrezeptoren
% -> externe Stimuli
% -> z.B. Speichelsekretion, Wuergereflex 


% Viszerale Schmerzen:
% -> starke Dehnung/Kontraktion ORganmuskulatur
% -> ueber spinale viszerale Afferenzen *nicht vagal)
% -> Rezeptoren in Serosa, am mesenterialansatz und *vermutlich) Organwaenden


% \end{frame}

% ************************ [33:31]

% \begin{frame}
% \frametitle{Viszerale Afferenzen}

% \begin{block}{Zellkoerper viszerale Afferenzen}

% Ganglion inferius/superius; N. vagus, thoralake, obere lumbale und sakrale Spinalganglien, Ganglion petrosum (arterielle Presso- und Chemorezeptoren)
% \end{block}


% \begin{block}{Axone viszerale Afferenzen}

% in Nn vagi, Nn splanchnici und glossopharyngeus


% - nahezu alle projizieren zum NTS
% - modulieren vegetative Reflexe auf Rueckenmarksebene
% - koenne Bestandteil grosser Reflexboegen sein
%     - MDT -> Rueckenmark/Medulla oblongata
%     - koennen vago-vagal sein
%     - koennen adnere Nerven mit einbeziehen
% - spinale Afferenzen oder Hirnnervenafferenz (v.a. vagal)

% \end{block}
% \end{frame}

% ************************ [35:36]

% Verbinudungen im Vegetativen Nervensystem

% [Abbildung: Maybe use this:  https://en.wikipedia.org/wiki/Autonomic_nervous_system#/media/File:Sistema_Nervioso_Autonomo.svg]



\section{Transmitter und Rezeptoren im Vegetativen Nervensystem}

\section{Effektororgane des vegetativen Nervensystems}




%% %% %% %% Review


\begin{frame}

 \frametitle{Jetzt* sollten Sie folgendes können}



\end{frame}




%% %% %% %% Feedbackhinweisblock

\begin{frame}
\frametitle{Danke für Ihr Feedback!}

\begin{columns}[c]

\begin{column}{6cm}
\begin{center}
 \includegraphics[width=\textwidth]{smilie_balloons.jpg}
\end{center}

\end{column}

\begin{column}{4cm}


\begin{center}
\includegraphics[width=\textwidth]{feedback_QR.png}
\end{center}
\end{column}


\end{columns}

\end{frame}



%% %% %% Bildnachweis
\begin{frame}
\frametitle{Bildnachweis}

\begin{tiny}

Teile dieser Vorlesung wurden übernommen von einer Vorlesung von Prof. Maike Glitsch, Medical School Hamburg. Wo nicht anders angegeben, stammen Abbildungen aus dieser Vorlesung.  Herzlichen Dank!


 
\begin{itemize}

\item
Achterbahn. Photo by \href{https://unsplash.com/@polarmermaid?utm_source=unsplash&utm_medium=referral&utm_content=creditCopyText}{Anne Nygård } on \href{https://unsplash.com/s/photos/roller-coaster?utm_source=unsplash&utm_medium=referral&utm_content=creditCopyText}{Unsplash}
  
\item
Darmgehirn – mehr als nur ein Bauchgefühl. Screenshot von einem Artikel von Luise Heine, NetDoktor, 2016.  \url{https://www.netdoktor.de/magazin/darmgehirn-mehr-als-nur-ein-bauchgefuehl/}

%% all lectures
\item
Luftballons mit frohen und traurigen Smilies. Photo by \href{https://unsplash.com/@artbyhybrid?utm_source=unsplash&utm_medium=referral&utm_content=creditCopyText}{Hybrid} on \href{https://unsplash.com/s/photos/feedback?utm_source=unsplash&utm_medium=referral&utm_content=creditCopyText}{Unsplash}
%%%%%%%%%%%

\end{itemize}
\end{tiny}
\end{frame}









\end{document}

%%% Frequently used snippets

%% \begin{columns}[c]

%% \begin{column}{5cm}
%% \end{column}

%% \begin{column}{5cm}
%% \end{column}


%% \end{columns}

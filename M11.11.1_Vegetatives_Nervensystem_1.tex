\documentclass{beamer}	
\mode<presentation>
 
\usepackage{pdfpages}
\usepackage{fancyvrb}
\usepackage{chemarr}

\usepackage{amsmath}		%% mathematics typesetting
\usepackage{amssymb}
 
\usepackage{epigraph}   %% nice setting of quotations

\usepackage{tabularx} %% allows to use row colours in tables

\usepackage{ulem}

\usepackage{booktabs}

\usepackage{siunitx} %% tpyeset SI units

\usepackage{CJKutf8} %% typeset Chinese characters

\usepackage{pdfpages}%% include pdfs


\usepackage{animate} %% show animated gifs

\DeclareMathAlphabet{\mathcalligra}{T1}{calligra}{m}{n}


% Color and Theme. Can be changed. However, this one's quite nice.
\usetheme{Madrid}
\definecolor{theme}{rgb}{0.84,0,0.21}
\usecolortheme[named=theme]{structure}


%%  Title information
\title[M11.11.1 Vegetatives Nervensystem I]{M11.11.1 Vegetatives Nervensystem 1}
\author[melanie.stefan@medicalschool-berlin.de]{}
\institute[]{Prof. Melanie Stefan - melanie.stefan@medcialschool-berlin.de}
\date{SoSe 2022}
 

% Table of contents to pop up at the beginning of each section
\AtBeginSection[]
{
  \begin{frame}<beamer>
    \frametitle{Outline}
    \tableofcontents[currentsection,currentsubsection]
  \end{frame}
}
 
\beamertemplatenavigationsymbolsempty

\begin{document}


{ \usebackgroundtemplate{\includegraphics[width=1.2\paperwidth]{MSB_Titelseite.pdf}} 
\begin{frame}

 \maketitle 

$\,$\\[6cm]


\end{frame} 
}



%% Hook: 


%% %% TLIA


%% %% Learning Objectives
 
\begin{frame}

 \frametitle{Nach dieser Vorlesung sollten Sie folgendes können}



\begin{block}{Grundlagen:}
\begin{itemize}
\item

\end{itemize}

\end{block}

\begin{block}{Klinik:}
\begin{itemize}
\item


\end{itemize}

\end{block}


\end{frame}




%% %% %% Main Body


\section{Vegetatives Nervensystem (VNS): Allgemeines }


% ************************
\begin{frame}{Vegetatives Nervensystem: Allgemeines}

\begin{itemize}
    \item 
auch: autonomes Nervensystem, viszerales Nervensystem
\item
zentrale und periphere Komponenten
\item
bekommt Anweisungen vom limbischen System
\end{itemize}

\end{frame}

% ***********************

\begin{frame}{Vegetatives Nervensystem: Allgemeines}

\begin{center}
\includegraphics[width=\textwidth]{brain_functions.png}    
\end{center}


\end{frame}



% ***********************

\begin{frame}{Vegetatives Nervensystem: Allgemeines}

\begin{itemize}
    \item 
    nicht willentlich beeinflussbar
    \item
    (schnelle) Anpassung vegetativer Funktionen des Körpers an aktuelle Erfordernisse, Koordination von Organfunktion: essenziell für Überleben
\end{itemize}



\end{frame}


\begin{frame}{Funktionen des vegetativen Nervensystem}

\begin{itemize}
    \item
    Unterstützung motorischer Leistung
    \item
    Körperhomöostase  (Aufrechterhaltung und Anpassung physiologischer Parameter)
    \item
    Vegetative Reflexe
    \item
    beeinflusst Aktivität von:
    \begin{itemize}
        \item 
        Erfolgsorganen  - Regulierung glatte Muskulatur
        \item
        Herzmuskulatur
        \item
        Drüsen und sekretorische Gewebe
    \end{itemize}
   
\end{itemize}

\end{frame}






\section{Komponenten des vegetativen Nervensystems}

\section{Transmitter und Rezeptoren im Vegetativen Nervensystem}

\section{Effektororgane des vegetativen Nervensystems}




%% %% %% %% Review


\begin{frame}

 \frametitle{Jetzt* sollten Sie folgendes können}



\end{frame}




%% %% %% %% Feedbackhinweisblock

\begin{frame}
\frametitle{Danke für Ihr Feedback!}

\begin{columns}[c]

\begin{column}{6cm}
\begin{center}
 \includegraphics[width=\textwidth]{smilie_balloons.jpg}
\end{center}

\end{column}

\begin{column}{4cm}


\begin{center}
\includegraphics[width=\textwidth]{feedback_QR.png}
\end{center}
\end{column}


\end{columns}

\end{frame}



%% %% %% Bildnachweis
\begin{frame}
\frametitle{Bildnachweis}

\begin{tiny}

Teile dieser Vorlesung wurden übernommen von einer Vorlesung von Prof. Maike Glitsch, Medical School Hamburg. Wo nicht anders angegeben, stammen Abbildungen aus dieser Vorlesung.  Herzlichen Dank!


 
\begin{itemize}

   


%% all lectures
\item
Luftballons mit frohen und traurigen Smilies. Photo by \href{https://unsplash.com/@artbyhybrid?utm_source=unsplash&utm_medium=referral&utm_content=creditCopyText}{Hybrid} on \href{https://unsplash.com/s/photos/feedback?utm_source=unsplash&utm_medium=referral&utm_content=creditCopyText}{Unsplash}
%%%%%%%%%%%

\end{itemize}
\end{tiny}
\end{frame}









\end{document}

%%% Frequently used snippets

%% \begin{columns}[c]

%% \begin{column}{5cm}
%% \end{column}

%% \begin{column}{5cm}
%% \end{column}


%% \end{columns}
